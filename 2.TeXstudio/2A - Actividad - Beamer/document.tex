\documentclass[12pt]{beamer}
\usepackage[utf8]{inputenc}
\usepackage[T1]{fontenc}
\usepackage{lmodern}
\usepackage[spanish]{babel}
\usepackage{graphicx}
\usetheme{Cuerna}
\usepackage{times}

\begin{document}
	\author{Giovanni Álvarez}
	\title{Informática Forense}
	\institute{Universidad Tecnológica Intercontinental}
	%\subtitle{}
	%\logo{}
	\date{03 de octubre del 2021}
	%\subject{}
	%\setbeamercovered{transparent}
	%\setbeamertemplate{navigation symbols}{}
	
	\begin{frame}[plain]
		\maketitle
	\end{frame}

	
\begin{frame}{Introducción}
	\begin{itemize}
\item El cómputo forense, también llamado INFORMATICA FORENSE, computación forense, análisis forense digital, examinación forense digital o Forensic es la aplicaión de técnicas científicas y analíticas especializadas en infraestructura tecnológica que permiten identificar, preservar, analizar y presentar datos que sean válidos dentro de un proceso legal.
\end{itemize}
\end{frame}

\begin{frame}
\frametitle{¿Qué es la informática forense?}

Hoy en día tenemos a nuestra disposición una serie de herramientas que protegen nuestra información personal en la red. Antivirus, software variado de seguridad y otro tipo de sistemas informáticos. Todas estas alternativas forman parte de una disciplina que conocemos como informática forense. Esta se encarga de adquirir, preservar y proteger datos procesados de forma electrónica y almacenados en un medio físico. Los sistemas de información son investigados, de forma periódica, para detectar cualquier pequeña vulnerabilidad que pueda poner en peligro la enorme cantidad de datos que se procesan y almacenan cada segundo.
%\tableofcontents
\end{frame}


\begin{frame}
\frametitle{¿Qué es lo que busca la informática forense?}

Es la disciplina que combina los elementos del derecho y la informática para recopilar y analizar datos de sistemas informáticos, redes, comunicaciones inalámbricas y dispositivos de almacenamiento de una manera que sea admisible como prueba en un tribunal de justicia.
\end{frame}


\begin{frame}
\frametitle{La importancia de la Informática Forense}

El papel que tiene la informática forense es principalmente preventivo y nos ayuda, mediante diferentes técnicas, a probar que los sistemas de seguridad que tenemos implementados son los adecuados para poder corregir errores y poder mejorar el sistema además de conseguir la elaboración de políticas de seguridad y la utilización de determinados sistemas para poder mejorar tanto el rendimiento como la seguridad del sistema de información.
Su objetivo es la investigación de sistemas de información para poder detectar cualquier clase de evidencia de vulnerabilidad que puedan tener. Asi mismo se persiguen diferentes objetivos de prevención, intentando anticiparse a lo que pudiera pasar así como establecer una solución rápida cuando las vulnerabilidades ya se han producido.
\end{frame}


\begin{frame}	
\frametitle{Imagénes}
\begin{figure}
	\centering
	\includegraphics[width=0.99\textheight]{../../../../Downloads/informatica-forense}
	%\caption{}
	\label{fig:informatica-forense}
\end{figure}
\end{frame}

\begin{frame}
\begin{figure}
	\centering
	\includegraphics[width=0.99\textheight]{"../../../../Downloads/informatica-forense (1)"}
	%\caption{}
	\label{fig:informatica-forense-1}
\end{figure}
	\end{frame}


\begin{frame}
	\frametitle{Vídeos}
	\framesubtitle{Hipervinculos}
	
	¿Qué es la informática forense?
	\textcolor{blue}{\url{https://www.youtube.com/watch?v=QXj37NPP5PU}}
	
	Así trabajan los forenses informáticos
	\textcolor{blue}{\url{https://www.youtube.com/watch?v=zr2ebs2lBII}}
	
	Herramientas Informatica Forense
	\textcolor{blue}{\url{https://www.youtube.com/watch?v=Tmb3R9ltkJw}}
\end{frame}


\begin{frame}
	\frametitle{Conclusión}
	\begin{itemize}
		
\item En conclusión, la utilización de la informática forense viene con una finalidad preventiva, esto puede ayudar  a las empresas para auditar mediante la práctica de diversas pruebas técinas, que los mecanismos de protección instalados y las condiciones de seguridad aplicadas a los sistemas de información son suficientes.
	\end{itemize}
\end{frame}


\begin{frame}
\frametitle{Enlaces}
\begin{itemize}
	
\item \textcolor{blue}{\url{https://escuelaselect.com/informatica-forense/}}

\item \textcolor{blue}{\url{https://www.sijufor.org/informacioacuten-relevante-en-materia-forense/la-importancia-de-la-informatica-forense}}

\item \textcolor{blue}{\url{https://protecciondatos-lopd.com/empresas/informatica-forense/}}

\end{itemize}
\end{frame}
\end{document}

