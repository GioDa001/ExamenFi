\documentclass[10pt,a4paper]{article}
\usepackage[latin1]{inputenc}
\usepackage[T1]{fontenc}
\usepackage{amsmath}
\usepackage{amsfonts}
\usepackage{amssymb}
\usepackage[spanish]{babel}
\usepackage{graphicx}
\author{Giovanni �lvarez}
\begin{document}
	
	%\title{Tablas}
	%\date{}
	%\maketitle
	
	\section{Ejercicio} % Secci�n 1
	
	Para obtener una simple tabla sin l�neas:
	
	\begin{tabular}{l c r} % Tabulaci�n y Divisor
		1 & 2 & 3 \\ 
		4 & 5 & 6 \\
		7 & 8 & 9 \\
	\end{tabular}

\section{Ejercicio} % Secci�n 2

Para obtener esta tabla en la que a�adimos algunas l�neas verticlaes:

\begin{tabular}{| l || c || r |} % Tabulaci�n y Divisor
	1 & 2 & 3 \\ 
	4 & 5 & 6 \\
	7 & 8 & 9 \\
\end{tabular}

\section{Ejercicio} % Secci�n 3

Ahora con l�neas horizontales: superior e inferior

\begin{tabular}{| l | c | r |} % Tabulaci�n y Divisor
	\hline % L�nea tan larga como la tabla
	1 & 2 & 3 \\ 
	4 & 5 & 6 \\
	7 & 8 & 9 \\
	\hline
\end{tabular}

\section{Ejercicio} % Secci�n 4

Ahora con l�neas horizontales: superior e inferior

\begin{center}
\begin{tabular}{| l | c | r |} % Tabulaci�n y Divisor
	\hline % L�nea tan larga como la tabla
	10 & 22 & 33\\ \hline
	44 & 55 & 66 \\ \hline
	77 & 88 & 99 \\
	\hline
\end{tabular}
\end{center}	
	
\end{document}